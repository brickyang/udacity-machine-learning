\documentclass[UTF8,nofonts]{ctexart}
\usepackage{xeCJK}
\setCJKmainfont[BoldFont=STHeiti,ItalicFont=STKaiti]{STSong}
\setCJKsansfont[BoldFont=STHeiti]{STXihei}
\setCJKmonofont{STFangsong}
\usepackage{amsmath}

\setlength{\parindent}{0pt}
\begin{document}
\textbf{Project 3:š 2.4 证明} \\

设有矩阵 $A$:
\begin{gather*}
A=
\begin{bmatrix} 
    1 & \cdots & 0 & x_{1c} & \cdots & x_{1n}  \\ 
    \vdots & \ddots & \vdots & \vdots & \ddots & \vdots \\
    0 & \cdots & 1 & x_{ic} & \cdots & x_{in} \\
    0 & \cdots & 0 & 0 & \cdots & \ y_{(i+1)n}\\
    \vdots & \ddots & \vdots & \vdots & \ddots & \vdots \\
    0 & \cdots & 0 & 0 & \cdots & y_{mn}\\
\end{bmatrix}
\end{gather*} \\
当 $A$ 的第 $c$ 列,即$x_{1c}$~$x_{ic}$ 全为 0 时,则有非零向量 $x$:
\begin{gather*}
x=
\begin{bmatrix} 
    0  \\ 
    \vdots \\
    x \\
    \vdots \\
    0 \\
\end{bmatrix}
\end{gather*}
即 $x$ 的第 $c$ 行为非零实数,其余各行均为 0,使得
\begin{gather*}
Ax=
\begin{bmatrix} 
    1 \\ 
    \vdots  \\
    0 \\
\end{bmatrix}
\cdot 0 + \cdots +
\begin{bmatrix} 
    x_{1c} \\ 
    \vdots  \\
    0 \\
\end{bmatrix}
\cdot e_{c} + \cdots + 
\begin{bmatrix} 
x_{1n} \\ 
\vdots  \\
y_{mn} \\
\end{bmatrix} 
\cdot 0 = 0 \\
\end{gather*} 


当 $A$ 的第 $c$ 列不全为 0 时,则有非零向量 $x$:
\begin{gather*}
x=
\begin{bmatrix} 
-x_{1c}  \\ 
\vdots \\
-x_{ic} \\
1 \\
\vdots \\
0 \\
\end{bmatrix}
\end{gather*}
使得
\begin{gather*}
Ax=
\begin{bmatrix} 
1 \\ 
\vdots  \\
0 \\
\vdots \\
0 \\
\end{bmatrix}
\cdot -x_{1c} + \cdots +
\begin{bmatrix} 
0 \\ 
\vdots  \\
1 \\
\vdots \\
0 \\
\end{bmatrix} 
\cdot -x_{ic} + 
\begin{bmatrix} 
x_{1c} \\ 
\vdots  \\
x_{ic} \\
\vdots \\
0 \\
\end{bmatrix}
\cdot 1 + \cdots + 
\begin{bmatrix} 
x_{1n} \\ 
\vdots  \\
x_{in} \\
\vdots \\
y_{mn} \\
\end{bmatrix} 
\cdot 0 \\ 
=  
\begin{bmatrix} 
-x_{1n} \\ 
\vdots \\
0 \\
\vdots  \\
0 \\
\end{bmatrix} 
+ \cdots + 
\begin{bmatrix} 
0 \\
\vdots \\
-x_{ic} \\ 
\vdots  \\
0 \\
\end{bmatrix} 
+ \cdots +
\begin{bmatrix} 
x_{1c} \\ 
\vdots  \\
x_{ic} \\
\vdots \\
0 \\
\end{bmatrix} 
+ \cdots +
\begin{bmatrix} 
0 \\ 
\vdots  \\
0 \\ 
\vdots  \\
0 \\
\end{bmatrix}
= 0 \\
\end{gather*}


即总是存在向量 $x$,使得 $Ax=0$ 成立,故矩阵 $A$ 不可逆,即 $A$ 为奇异矩阵。\\


\textbf{补证} \\

当 $Ax=0$ 成立时,其中 $x$ 为非零向量,若 $A$ 可逆,则有
\begin{equation*}
A^{-1}Ax=Ix=x=0
\end{equation*}
与前提 $x$ 为非零向量矛盾,故 $A$ 不可逆。
\end{document}
